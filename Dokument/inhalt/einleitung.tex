\chapter{Einleitung}
\label{Kapitel:Einleitung}


\section{Motivation}
 
 Die Menge an analysierbaren  Informationen und den dazugehörigen Daten in Unternehmen nehmen immer stärker zu.  Durch den Zuwachs an Daten wird es aber auch immer schwieriger diese effektiv auszuwerten. Die  Daten in einzelnen und unterschiedlichen Programmen zu verwalten und zwischen diesen auszutauschen lässt sich zeitlich nicht mehr bewerkstelligen. Außerdem sind einige Datenquellen so stark angewachsen, das sie sich mit herkömmlichen Tabellenkalkulationsprogrammen nicht mehr performant auswerten lassen. \\
Diesem Problem hat sich die Firma SAP gestellt. Die von ihnen entwickelte Data Warehousing Workbench it ein Baustein der gesamten Software Suite.  Durch diese Zusammenstellung kann in einem  Unternehmen die gesamte Datenbasis unbearbeitet oder aufbereitet für jeden zur Verfügung gestellt werden. So können Informationen direkt in verschiedenen Abteilungen erfasst und ausgewertet werden. Dazu gehören auch komplexere Data Mining Prozesse, durch die bestimmte Muster in den Datensätzen gefunden werden können, die beispielsweise auf versteckte Kundenanforderungen ,,aufdecken".  Dies ist mittlerweile notwendig, um am Mark konkurrenzfähig zu bleiben. \cite[S. 46 f.]{Herschel:2013kz}
 
\section{ Aufbau der Arbeit}

Zunächst wird der Begriff ,,Business Intelligence`` abgesteckt, um einen besseren Einblick in das Themengebiet zu erhalten. Dafür wird zusätzlich zur Definition  eine aufgeschlüsselte Übersicht über den Themenbereich gegeben. Im Anschluss daran werden die einzelnen Module der Data Warehousing Workbench im Detail beschrieben. Abschließend folgen diverse Anleitungen die die Administration der Workbench in SAP BW 7.0 bildlich sowie textuell darstellen.


 
