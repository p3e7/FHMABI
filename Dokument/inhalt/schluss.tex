\chapter{Abschlussbemerkung und Ausblick}


\section{Fazit} 
\label{Abschnitt:}
Die für die Usability und User Experience beschrieben Kriterien konnten so erweitert werden, das sie mit Videospielelementen in Verbindung gebracht werden konnten. So konnte ein Kriterien Katalog für die User Experience erstellt werden. Es wurde eine valide Usability und User Experience Evaluation so ausgearbeitet, das das Spiel Æthershards evaluiert werden konnte. In diesem Rahmen ist ein neuer Fragebogen entstanden. Er kann sowohl auf Spielprototypen, als auch auf Spiele in späteren Entwicklungsphasen angewendet werden, um die User Experience zu evaluieren. Zusätzlich ist ein anderer Fragebogen entstanden, mit dessen Hilfe eine Spielidee mit geringer User Experience erkannt werden konnte. Dieser Fragebogen kann aber nicht ohne Veränderung auf andere Spiele, Spielideen oder Spielkonzepte übertragen werden. 


\section{Ausblick}
\label{Abschnitt:}
Das Ziel der Arbeit wurde aber vollständig erreicht. Einsteigern in der Spieleentwicklung wird durch diese Arbeit ein Hilfsmittel an die Hand gegeben. Die erarbeiteten Kriterien sind hinreichend aufgeführt. Es aber gibt weitere Evaluationsmethoden. Auch die Kriterien können anders formuliert und somit feiner abgesteckt werden. Dies geht weiter in die Tiefe und ist sinnvoll. Damit wären die Kriterien aber ggf. nur noch von Spieleentwicklern mit mehr Erfahrung nutzbar. \\
Da vermutet wurde, das ein Pen and Paper Prototyp die User Experience beeinflussen kann ist es wichtig weitere Tests mit verschiedenen Prototypen und fertigen Spielen durchzuführen um zu ermitteln, wie stark die Abweichungen sind. 