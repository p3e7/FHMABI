
%%% Doc: ftp://tug.ctan.org/pub/tex-archive/macros/latex/required/babel/babel.pdf
% Languagesetting
\usepackage{babel}	% Sprache

\usepackage{textpos} 


\usepackage{fixltx2e}	% Verbessert einige Kernkompetenzen von LaTeX2e
\usepackage{ellipsis}	% Korrigiert den Wei�raum um Auslassungspunkte

\usepackage{placeins} 

\usepackage{ifpdf}
\ifpdf
\pdfinfo {
	/Author (\theauthor)
	/Title (\thetitle)
	/Subject ()
	/Keywords ()
%	/CreationDate (D:YYYYMMTTHHMMSS)
}
\fi



%%% Doc: www.cs.brown.edu/system/software/latex/doc/calc.pdf
% Calculation with LaTeX
\usepackage{calc}

%%% Doc: ftp://tug.ctan.org/pub/tex-archive/macros/latex/contrib/xcolor/xcolor.pdf
% Farben
% Incompatible: Do not load when using pstricks !
\usepackage[
	table % Load for using rowcolors command in tables
]{xcolor}

\usepackage{tikz}
\usetikzlibrary{% 
   arrows,% 
   calc,% 
   fit,% 
   patterns,% 
   plotmarks,% 
   shapes.geometric,% 
   shapes.misc,% 
   shapes.symbols,% 
   shapes.arrows,% 
   shapes.callouts,% 
   shapes.multipart,% 
   shapes.gates.logic.US,% 
   shapes.gates.logic.IEC,% 
   er,% 
   automata,% 
   backgrounds,% 
   chains,% 
   topaths,% 
   trees,% 
   petri,% 
   mindmap,% 
   matrix,% 
   calendar,% 
   folding,% 
   fadings,% 
   through,% 
   positioning,% 
   scopes,% 
   decorations.fractals,% 
   decorations.shapes,% 
   decorations.text,% 
   decorations.pathmorphing,% 
   decorations.pathreplacing,% 
   decorations.footprints,% 
   decorations.markings,% 
   shadows} 
   
%%% Doc: ftp://tug.ctan.org/pub/tex-archive/macros/latex/required/graphics/grfguide.pdf
% Bilder
\usepackage[%
	%final,
	%draft % do not include images (faster)
]{graphicx}


% bessere Abstaende innerhalb der Tabelle (Layout))
% -------------------------------------------------
%%% Doc: ftp://tug.ctan.org/pub/tex-archive/macros/latex/contrib/booktabs/booktabs.pdf
\usepackage{booktabs}


%%% Doc: ftp://tug.ctan.org/pub/tex-archive/macros/latex/contrib/enumitem/enumitem.pdf
% Better than 'paralist' and 'enumerate' because it uses a keyvalue interface!
% Do not load together with enumerate.
%\usepackage{enumitem}

\usepackage{paralist}


%%% Doc: http://www.ctan.org/tex-archive/macros/latex/contrib/acronym/acronym.pdf
% Usage:
%        Definition: \acro{ acronym }[ short name ]{ full name }
%        Nutzung im Text: \ac{acronym}
 \usepackage[
 	footnote,	% Full names appear in the footnote
 	%smaller,		% Print acronym in smaller fontsize
 	%printonlyused %
 ]{acronym}
%\chapter*{Abkürzungsverzeichnis}
\begin{acronym}[BiPRO ] %Längster Begriff
\setlength{\itemsep}{-\parsep}
    \acro{LBG}{Location-based Game}
    \acro{UX}{User Experience}
    \acro{MMORPG} {Massively Multiplayer Online Role-Playing Game}
    \acro{EA}{Electronic Arts}
	\acro{GAAP}{Generally accepted accounting principles}
\end{acronym}



%% Kopf und Fusszeilen====================================================
%%% Doc: ftp://tug.ctan.org/pub/tex-archive/macros/latex/contrib/koma-script/scrguide.pdf
\usepackage[%
   automark,         % automatische Aktualisierung der Kolumnentitel
   %nouppercase,      % Grossbuchstaben verhindern
   %markuppercase    % Grossbuchstaben erzwingen
   %markusedcase     % vordefinierten Stil beibehalten
   %komastyle,       % Stil von Koma Script
   %standardstyle,   % Stil der Standardklassen
]{scrpage2}



%% UeberSchriften (Chapter und Sections) =================================
% -- Ueberschriften komlett Umdefinieren --
%%% Doc: ftp://tug.ctan.org/pub/tex-archive/macros/latex/contrib/titlesec/titlesec.pdf
\usepackage{titlesec}

% -- Section Aussehen veraendern --
% --------------------------------
%% -> Section mit Unterstrich
% \titleformat{\section}
%   [hang]%[frame]display
%   {\usekomafont{sectioning}\Large}
%  {\thesection}
%   {6pt}
%   {}
%   [\titlerule \vspace{0.5\baselineskip}]
% --------------------------------

% -- Chapter Aussehen veraendern --
% --------------------------------
\titleformat{\chapter}[block]	% {command}[shape]
  {\usekomafont{chapter}\huge\sffamily\bfseries}	% format
  {   										% label
      {\thechapter.} \filright%
  }%}
  {1pt}										% sep (from chapternumber)
  {\vspace{0.5pc} \filright {}}   % {before}[after] (before chaptertitle and after)
  [\vspace{0.5pc} \filright {}]

% \titleformat{\chapter}[]%
%    {\usekomafont{chapter}\huge\sffamily\bfseries}%
%    {\thechapter}%
%    {1em}%
%    {}%


\usepackage{rotating}


\usepackage[numbers,square]{natbib}
%\usepackage{cite}
%\bibliographystyle{dinat}
%\citestyle{alpha}
\bibliographystyle{alphadin}


% Quotes =================================================================
%% Doc: ftp://tug.ctan.org/pub/tex-archive/macros/latex/contrib/csquotes/csquotes.pdf
% Advanced features for clever quotations
\usepackage[%
   babel,            % the style of all quotation marks will be adapted
                     % to the document language as chosen by 'babel'
   german=quotes,		% Styles of quotes in each language
   %german=guillemets,
   english=british,
   french=guillemets
]{csquotes}
\usepackage{floatflt}

\usepackage{wrapfig}

%\usepackage{subfigure}

\usepackage{blindtext}

\usepackage{listings}
\lstset{language=html}
\definecolor{lila}{RGB}{112, 6, 147}
\definecolor{kommentgreen}{RGB}{5,132,71}
\definecolor{grey}{RGB}{242,242,242}  
\definecolor{darkgreen}{named}{green}
\definecolor{darkblue}{named}{blue}
\definecolor{lightblue}{RGB} {63,95,191}
\definecolor{darkred}{named}{red}
\definecolor{grau}{named}{gray}
\definecolor{fh_orange}{rgb}{0.953,0.201,0}
\definecolor{fh_grau}{rgb}{0.76,0.75,0.76}
\definecolor{light_green}{RGB}{199,199,199}

\definecolor{listinggray}{gray}{0.9}
\definecolor{lbcolor}{rgb}{0.9,0.9,0.9}

\lstset{
	tabsize=3,
	float=tbph,
	frame=single,
	extendedchars,
	breaklines=true,
	basicstyle=\fontsize{9pt}{9pt}\selectfont,
	columns=flexible, %ist notwendig, damit man Quelltext aus den Listings kopieren kann
	numbers=left, 
	numberstyle=\color{black},
	captionpos=b,
	aboveskip=7mm,
	backgroundcolor=\color{grey}
}

\lstdefinestyle{java}
{
	language=Java,
	keywordstyle=\color{lila},  	% underlined bold black keywords 
	identifierstyle=\color{blue}, 
	commentstyle=\color{kommentgreen}, % white comments 
	stringstyle=\color{black},
}

\lstdefinestyle{xml}
{
	language=xml,
	basicstyle=\fontsize{9pt}{9pt}\selectfont\color{black},
	keywordstyle=\color{lila},  	% underlined bold black keywords 
	%Hier können bei Bedarf noch weitere Keywords eingetragen werden
	keywords={name, value, version, encoding, id, type, xmlns:xsi, ref, namespace,bachelorarbeit,thema,vorname,note},
	identifierstyle=\color{black},  
	stringstyle=\color{blue},  
	commentstyle=\color{lightblue},
	morecomment=[s]{<!--}{-->},
	rulecolor=\color{black}
}


\colorlet{punct}{red!60!black}
\definecolor{background}{HTML}{EEEEEE}
\definecolor{delim}{RGB}{20,105,176}
\colorlet{numb}{magenta!60!black}



\lstdefinelanguage{json}{
    basicstyle=\fontsize{9pt}{9pt}\selectfont\color{black},
    literate=
     *{0}{{{\color{numb}0}}}{1}
      {1}{{{\color{black}1}}}{1}
      {2}{{{\color{numb}2}}}{1}
      {3}{{{\color{black}3}}}{1}
      {4}{{{\color{numb}4}}}{1}
      {5}{{{\color{numb}5}}}{1}
      {6}{{{\color{numb}6}}}{1}
      {7}{{{\color{numb}7}}}{1}
      {8}{{{\color{numb}8}}}{1}
      {9}{{{\color{numb}9}}}{1}
      {:}{{{\color{punct}{:}}}}{1}
      {,}{{{\color{punct}{,}}}}{1}
      {[}{{{\color{delim}{[}}}}{1}
      {]}{{{\color{delim}{]}}}}{1},
      keywordstyle=\color{lila},  	% underlined bold black keywords 
	%Hier können bei Bedarf noch weitere Keywords eingetragen werden
	keywords={name, value, version, encoding, id, type, xmlns:xsi, ref, namespace,bachelorarbeit,thema,vorname,note,},
}


% Taken from Lena Herrmann at 
% http://lenaherrmann.net/2010/05/20/javascript-syntax-highlighting-in-the-latex-listings-package
%\documentclass{article}
%\usepackage{listings}
%\usepackage{color}
\definecolor{lightgray}{rgb}{.9,.9,.9}
\definecolor{darkgray}{rgb}{.4,.4,.4}
\definecolor{purple}{rgb}{0.65, 0.12, 0.82}

\lstdefinelanguage{JavaScript}{
  keywords={typeof, new, true, false, catch, function, return, null, catch, switch, var, if, in, while, do, else, case, break},
  keywordstyle=\color{blue}\bfseries,
  ndkeywords={class, export, boolean, throw, implements, import, this},
  ndkeywordstyle=\color{darkgray}\bfseries,
  identifierstyle=\color{black},
  sensitive=false,
  comment=[l]{//},
  morecomment=[s]{/*}{*/},
  commentstyle=\color{purple}\ttfamily,
  stringstyle=\color{red}\ttfamily,
  morestring=[b]',
  morestring=[b]"
}

\lstset{
   language=JavaScript,
   backgroundcolor=\color{lightgray},
   extendedchars=true,
   basicstyle=\footnotesize\ttfamily,
   showstringspaces=false,
   showspaces=false,
   numbers=left,
   numberstyle=\footnotesize,
   numbersep=9pt,
   tabsize=2,
   breaklines=true,
   showtabs=false,
   captionpos=b
}

\lstset{literate=%
{Ö}{{\"O}}1
{Ä}{{\"A}}1
{Ü}{{\"U}}1
{ß}{{\ss}}2
{ü}{{\"u}}1
{ä}{{\"a}}1
{ö}{{\"o}}1
}


\usepackage{multicol}

\usepackage{nameref}

\usepackage{tabularx}

\usepackage{subfigure}

\usepackage{hyperref}
\hypersetup{breaklinks=true}
\hypersetup{colorlinks=true,linkcolor=black,urlcolor=black,citecolor=black}
%\hypersetup{frenchlinks}	% Use small caps instead of color for links
%\hypersetup{pdfpagemode=FullScreen}
%\hypersetup{pdfstartpage=3}
%\hypersetup{pdfstartview=Fit}


% Tabellen ueber mehere Seiten
% ----------------------------
%%% Doc: ftp://tug.ctan.org/pub/tex-archive/macros/latex/contrib/carlisle/ltxtable.pdf
% \usepackage{ltxtable} % Longtable + tabularx
                        % (multi-page tables) + (auto-sized columns in a fixed width table)
% -> nach hyperref laden
%\usepackage{ltxtable}
%\usepackage{longtable}
\usepackage{tabulary}


% Schusterjunge und Hurenkinder verhindern
\clubpenalty=1000
\widowpenalty=1000
\displaywidowpenalty=1000

% Trennen von Bindestrich oder so ...
%\defaulthyphenchar=127

\usepackage[T1]{fontenc}
\usepackage{alltt}
\usepackage{marvosym}
\usepackage{fancybox}
\usepackage[hang,small,bf]{caption}
\usepackage{float} 
\usepackage{multirow}
\usepackage{pgfplots}
\usepackage{tikz}
\usepackage{pdfpages} 