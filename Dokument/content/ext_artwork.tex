\section{Artwork}
\label{sec:ext_artwork}


\subsection{2D/3D} 
%\label{sec:}
Die Wahl zwischen 2D und 3D ist nicht einfach. Auf der einen Seite spricht die leichte Bedienung und die um ein vielfaches einfachere Erstellung von Spielelementen. Auf der anderen Seite überfordert einen ein 2D Spiel nicht so schnell wie ein 3D Spiel mit zu komplexen Bedienelementen. In der Vergangenheit konnte man zum Beispiel beim Wechsel von 2D zu 3D Spielekonsolen sehr schön sehen, wie die Controller um ein vielfaches komplexer geworden sind. Allein für die Kamera benötigt man meist schon ein „Analog-Stick“. Da sich diese Arbeit jedoch im Bereich der touchbased games bewegt, muss man für jedes Bedienelement Bildschirmplatz Aufgeben, somit kann der Spieler immer weniger vom Spielgeschehen sehen. 
Auf der anderen Seite kann es bei 2D Spielen schnell zu Komplikationen kommen, wenn der Hintergrund mit dem Vordergrund verschmilzt. Einige „Wimmelbild“-Spiele nutzen dieses Prinzip, um es dem Benutzer zur Aufgabe zu machen im Hintergrund versteckte Gegenstände zu finden.  Ein weiterer Punkt sind die Zeichenprogramme. Mit Paint oder ähnlichen Malprogrammen kann man schnell und unkompliziert Grafiken und kleine animationen erstellen ohne große Kenntnisse über das jeweilige Programm zu haben.
Nimmt auf der anderen Seite 3D modellierungsprogramme hinzu, erschweren sie den workflow erheblich. Es gibt unheimlich viele exportformate die dann auch wieder importiert werden können und müssen.  Zudem werden meiste eine oder mehrere Vorlagezeichnungen verwendet, um überhaut ein 3D Modell erstellen zu können.
Auf der anderen Seite bieten 3D Spiele mehr immersion und die entwicklung wird derzeit auch mit vielen kostenlosen tools, wie blender, vereinfacht. Für kommerzielle Produkte aus diesem Bereich fallen oft mehrere tausend Euro an Kosten an. 
Dazu werden auch mehrere Entwickler benötigt.

\subsection{Animation} 
%\label{sec:}
Auf Grund der Entscheidung die Grafiken im 2D Bereich zu belassen beschränkt sich dieses Kapitel lediglich auf die Animation von zweidimensionalen Bildern.

9.2	Arten der Animation 

Mehrere Bilder
Sprites 
Animationen per Skelett
Bildmanipulation

\url{http://dl.acm.org/citation.cfm?id=192186}