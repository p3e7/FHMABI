\chapter{Einleitung}
\label{cha:einleitung}

%\section{Überblick} 
%\label{sec:uberblick}
Diese Arbeit beschreibt die Konzeption und Entwicklung eines Location-based Games. Dies beinhaltet einen genaueren Einblick in die verschiedenen Phasen der Spieleentwicklung. Sowie eine auf Kriterien basierte Evaluation nach Niklas, der Tools und Technologien die zur Entwicklung eines solchen Spiels genutzt werden können. 

\section{Motivation}
\label{sec:motivation}
Unter der Prämisse nach Abschluss des Studiums als Gamedesigner zu arbeiten und da in der Fachliteratur immer wieder zu lesen ist, auf wie vielen unterschiedlichen Wegen der Einstieg in die Spieleindustrie möglich sei, haben wir beschlossen, einen dieser Wege zu gehen und zu dokumentieren. Des Weiteren gibt es auf Grund des Facettenreichtums der Spiele und ihrer Entwickler Unmengen an Büchern, die sich Quantitativ und Qualitativ sehr stark unterscheiden. So wie es sich mit den Quellen verhält, ist es auch mit den Entwicklern. Ein Großteil der Spieleentwickler sind Quereinsteiger. Der bis Dato einzige Versuch, den Einstieg in die Spielentwicklung zu generalisieren wurde von der Games Academy gestartet („Der direkte Einstieg in die Branche ist gleichzeitig Ziel und Maßstab dieser Ausbildung.“ Sabrina Wanie, Creative Director, DIGITRICK) 1.\\
Um einen möglichst guten Überblick zu bekommen haben wir ein Location-based Game als zu entwickelndes Spiel ausgewählt, da hierfür sehr viele unterschiedliche Technologien benötigt werden. Auf der einen Seite gibt das Spiel mit seinen Spielmechaniken, bei denen viele Feinheiten beachtet werden müssen, zum anderen muss man die Server-Client-Kommunikation beachten damit das Spiel überhaupt funktionstüchtig ist. \\

Dazu kommt die neue Funktionalität der Location-based Games. Denn die Positionsbestimmung bietet nicht nur eine weitere Eingabemöglichkeit, sondern lässt sich auch auf verschiedenste Arten auswerten. So kann man daraus Bewegungsmuster erhalten oder Heatmaps über die Anzahl der Menschen bzw. Smartphones an einem Ort erstellen. Location-based Games sind nun schon seit einiger Zeit auf dem Mark, sie haben aber sehr viel weniger Entwicklungsphasen durchlebt, wie zum Beispiel Strategiespiele. Somit bietet sich hier die Möglichkeit ein ganz neues Spiel zu entwickeln, welches sich aus der Masse der verfügbaren Spiele abhebt.
Aber auch die Wahl der richtigen Werkzeuge, wie die Entwicklungsumgebung, Programmiersprache und Technologien fallen hierbei zum Teil schwer ins Gewicht, da jede Entscheidung Vor- und Nachteile mit sich bringt. \\
Ein weiterer, für uns wichtiger, Aspekt ist,  dass es durch die Entwicklung eines Location-based Games möglich ist die jeweiligen Spieler ins Freie zu bringen. Denn üblicherweise fesselt ein Großteil der verfügbaren Videospiele die Spieler vor den Bildschirmen zu Hause. Dies macht zwar den Charme einiger Spiele aus, hat aber dazu geführt, dass es bei vielen Menschen zu einem immer größeren Bewegungsmangel2 kommt. 

\section{Zielsetzung}
\label{sec:zielsetzung}
Diese Arbeit hat das Ziel, einen simplen Prototypen, für ein von uns definiertes  Location-based Game, zu entwickeln und dessen Entwicklungsvorgang (möglichst genau) zu beschreiben. Dieser Prototyp wird kein vollständiges oder spielbares Spiel sein, sondern soll lediglich zeigen, dass eine von uns ausgearbeitete Spielidee mit den von uns gewählten Werkzeugen relativ simpel umgesetzt werden kann. Dies beinhaltet die Ermittlung einer oder mehrerer geeigneter Plattformen, Programmiersprachen und Entwicklungsumgebungen. Des Weiteren wird eine Spielidee von Grund auf entwickelt und so präzise wie möglich geschildert, um den Ablauf der Entwicklung  zu dokumentieren.