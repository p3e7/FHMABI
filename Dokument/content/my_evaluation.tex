\chapter{Kriterien basierte Evaluation für Videospiele}
\label{cha:my_evaluation}
%Das folgende Kapitel beschreibt, wie und welchem Rahmen Gameplayaspekte dienen um den Spielspaß positiv oder auch negativ zu beeinflussen. Somit lässt sich die Güte eines Spiels ermitteln.

Im folgenden Kapitel wird beschrieben, wie unter zur Hilfenahme der UX-Aspekte ein Spiel bewertet werden kann. 


\section{Definition von Kriterien und deren Gewichtung} 
%\label{sec:}

\subsection{Verfügbarkeit 1-5} 
%\label{sec:}

Einer der wichtigsten Punkte ist die Verfügbarkeit. Ein Spiel, das auf vielen Geräten gespielt werden kann ermöglicht es, dass viel mehr Spieler ein Spiel spielen kann.
Daher setzen gerade in der Indie-Szene viele Entwickler auf Plattformunabhängigkeit. Bei großen Spieleherstellern ist dies jedoch wieder etwas anders. So haben alle Spielekonsolen ihre exklusiv Titel.
Gerade bei Nintendo ist dies sehr ausgeprägt. Dadurch das sie nur Spiele für ihre eigenen Konsolen entwickeln, führt dies dazu, dass Spieler die Konsolen kaufen müssen, die ist aber auch ein etwas umfangreicheres Thema und es müsste tiefer auf die Marketing Aspekte eingegangen werden. Daher ist für die Bewertung zu sagen, je größer die Verfügbarkeit desto besser.

\subsection{Tension Release 1-10} 
%\label{sec:}

Wie in Kapitel §x vorgestellt ist Tension Release der Punkt, um den sich der Spielspaß hauptsächlich dreht. 

\subsection{Progress 1-5} 
%\label{sec:}

Zu Progress gehört zum Beispiel ein Levelsystem, sowie Achievments. Denn der Spieler soll sehen, wie weit er bereits fortgeschritten ist. Zudem soll es motivieren, da dem spieler die ganze Zeit vor Augen gehalten werden soll, wie weit er ist und was er noch tun muss um das nächste lvl zu erreichen oder den nächsten Punkt zu bekommen. Dies kann z.B. auch durch besser werdende Items erzielt werden. Dadurch kann im besten bzw. schlimmsten Fall sogar eine so genannte Suchtspirale entstehen. "Nur noch 3 punkte bis zum nächsten lvl?! Lvl up! Ui jetzt noch einmal 5 das mach ich auch eben noch"

\subsection{Umfang 1-3} 
%\label{sec:}

Der Umfang eines Spiels beinhaltet vieles zum einen kann dazu die Geschichte gehören zum anderen aber auch die verschiedenen Spiel Elemente. auch beim umfang gibt es eine wie in §x beschrieben eine goldene menge damit der spieler nicht erschlagen wird oder nach nur 5 Minuten schon alles gesehen hat.

\subsection{Dichte/Atmosphäre 1-3} 
%\label{sec:}

durch die Atmosphäre wird der spieler tiefer in die Spielwelt eingebunden. sie ist auch ein nicht zu vergessendes Stilmittel, kann aber auch je nach genre ganz anders aufgefasst werden. sie kann aber auch dabei helfen, ein spiel völlig anders darzustellen. nimmt man z.B. ein Sudoku und fügt statt der zahlen 1-9, 9 verschieden farbige steine, wirkt das ganze trotz gleicher Mechanik völlig anders.

\subsection{Social 1-2} 
%\label{sec:}

Soziale Aspekte sind ein zweischneidiges Schwert. Spieler legen wert darauf ihre Spielerfahrungen mit anderen zu teilen. Dadurch sind heutzutage völlig neue Kanäle der Kommunikation entstanden. Sieht man sich die Webseite twitch.tv an, gibt es dort eine Vielzahl von Spielern beim spielen zu schauen. durch die sogenannten lets plays verdienen mittlerweile immer mehr Menschen Geld. Auf der anderen Seite gibt es die von so gut wie allen Spielern als nervig empfundenen Facebook anfragen

\subsection{Preis 1-2} 
%\label{sec:}

Das Thema Preis ist im Rahmen dieser Arbeit eher ein Randthema, daher wird versucht, die wichtigsten Eckpunkte aufzugreifen um einen groben Überblick zu erhalten. Es lässt sich aber erkennen, dass Spieler gerne zu free 2 play Titeln tendieren, da man bei kostenlosen Spielen nichts falsch machen kann. Aber seit der Einführung der apps, hat sich generell einiges am Preismodell vieler Spiele geändert. denn durch einen geringeren Einstiegspreis vergrößert sich auch die Verfügbarkeit.(GENAUER ERKLÄREN!)

\section{Zu wertende Spiele} 
%\label{sec:}

\subsection{Mobbles} 
%\label{sec:}

\subsection{Scriggeles} 
%\label{sec:}

\subsection{Æthershards Version 1} 
%\label{sec:}

\subsection{Æthershards Version 2} 
%\label{sec:}

\section{Auswertung} 
%\label{sec:}

