\chapter{Einleitung}
\label{cha:einleitung}

\section{Motivation}
\label{sec:motivation}
 
Why should we define the term videogame? Because we have reasons to study videogames. What are these reasons? James Newman gives us an answer:
“While scholars identify a range of social, cultural, economic, political and technological factors that suggest the need for a (re)consideration of videogames by students of media, culture and technology, here, it is useful to briefly examine just three reasons why videogames demand to be treated seriously: the size of the videogames industry; the popularity of videogames; videogames as an example of human-computer interaction.” 


Before discussing heuristics for video games, we want to get a clear understanding of the terminology “video game“. Esposito provides an interesting definition for a video game:
“A video game is a game which we play thanks to an audiovisual apparatus and which can be based on a story” (Esposito 2005)

\cite{Nicolas:tk}

Fast alle Videospiele sind so konzipiert, dass die Spieler sich vor einem Bildschirm befinden müssen um das jeweilige Spiel spielen zu können. 

Location-based Games sind Videospiele, in denen die geographische Position Spielers mit in das Spielgeschehen einfließt.
Dies bringt neue Interaktionsmöglichkeiten mit sich. 



Ein Großteil der für Smartphones erhältlichen Videospiele sind Portierungen von bereits auf anderen Systemen erhältlichen Spielen. Diese bereits veröffentlichten Spiele wurden in den meisten fällen lediglich um die Fähigkeit, mit Touchscreen eingaben arbeiten zu können, erweitert.
Mittlerweile gibt es jedoch auch Spiele, die exklusiv für Smartphones bzw. Tablets entwickelt werden. Diese werden in der Regel auch meistens für Konsolen parallel entwickelt und bieten daher kaum neue Inhalte, die den Spieler reizen könnten. In diesem Kapitel wird daher ein anderer Ansatz beschrieben, der bei der Entwicklung eines Location-based Game zum Einsatz kommt und somit gewährleistet wird, dass diese Art von Spielen neue Inhalte dem Spieler bieten.

\url{http://dl.acm.org/citation.cfm?id=1111302}

\section{Zielsetzung}
\label{sec:zielsetzung}


- Was ist möglich in einem LBG\\
- Was verbirgt sich hinter dem Thema User Experience\\
- Was sind die Kernelemente eines Videospiels im Hinblick auf die UX\\
- Wie gut lässt sich die UX eines Spiels bewerten\\


\section{Aufbau der Arbeit}
\label{sec:aufbauDArbeit}

Das zentrale Thema dieser Arbeit ist der Begriff User Experience. Doch um einen Einstieg in dieses Thema zu bekommen, wird in den ersten Kapiteln ein Überblick über den Hintergrund und Spielkonzepte von Loaction-based Games gegeben. Darauf werden mehrere Genrevertreter betrachtet. Darunter befindet sich auch ein Prototyp für ein LBG, Æthershards, der in der vorangegangenen Projektarbeit erstellt wurde. \\
Da dieses Spiel noch nicht fertig entwickelt worden ist bietet es sich an die Einzelnen Spielelemente genauer zu betrachten, da schon bei der ursprünglichen Definition der Spielmechaniken einige nur mit bedenken festgelegt wurden. Diese Bedenken wurden durch eine Umfrage bestärkt. 
Somit ist es nun wichtig die einzelne Faktoren, die die User Experience ausmachen herauszugreifen und zu definieren. Mit diesen Faktoren können dann die zuvor beschriebenen Spiele bewertet werden. Auf dieser Basis kann dann die Spielidee von Æthershards weiterentwickelt werden.
%\section{Related Work}
%\label{sec:relatedWork}
%
%- L A Lehmann. Location-based Mobile Games\\
%- Mögliche Spielmechaniken \\
%- Nicht nur zur Unterhaltung interessant \\
%- Sponsored Locations \\
%- Möglichkeiten zur Positionsbestimmung
