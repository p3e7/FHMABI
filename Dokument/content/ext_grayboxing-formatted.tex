\chapter{Grayboxing}
\label{cha:ext_grayboxing}

Um Zeit und kosten zu sparen, kann man das so genannte Grayboxing einsetzen. Diese Technik bedeutet, dass bestimmte Teile eines Spiels, durch stark vereinfachte Modelle des finalen Modells eingesetzt werden. Oftmals werden in den ersten Testphasen eines Spiels untexturierte Level oder Levelelemente benutzt um zu schauen, in wie fern ein Level überhaupt spielbar ist. Man kann diesen Begriff aber auch ausdehnen und so zum Beispiel die Musik stark reduzieren oder ggf. ganz entfernen. Auch Charaktermodelle können durch einfache Quader oder Kugeln ersetzt werden. Somit kann schon zu Anfang der Entwicklungszeit relativ schnell Spielmechaniken getestet werden. Jedoch sollte hierbei darauf geachtet werden, dass lediglich Mechaniken getestet werden können. Spiele die stark auf Kunst, gemeint sind Visualisierung und Klangbild, ausgelegt sind, sind wesentlich schwieriger bis gar nicht "Graybox-bar", Da durch den Artstyle gezielt bestimmte Emotionen hervorgerufen werden sollen. \\ Stellt man sich das Anhand zweier Beispiele vor, wird das es deutlicher. Nimmt man ein Strategiespiel zur Hand, sind die wichtigsten Elemente der Kampf und der Aufbau. In beiden Szenarien kommt es im Grunde nur auf die implementierten Spielmechaniken an. Funktionieren alle Bounding-Box checks auch wenn sich 100 Einheiten auf dem Bildschirm befinden? Lassen sich Einheiten und Gebäude nach dem vorgegeben Schema aufrüsten? \\ Nimmt man hingegen ein Horror-Spiel, das durch seine beklemmende Stimmung dem Spieler einflößen soll, lässt sich dies nicht herausfinden. Denn der Spieler wird sich anders fühlen, wenn er in einem untexturierten Raum vor einem großen Würfel steht oder der virtuelle Pulsschlag aus den Boxen ertönt, die Musik unterbrochen wird und vor ihm ein Monster steht, das direkt aus einem seiner wildesten Alpträume entsprungen sein könnte. (1)[Seite 300]1.	Sylvester T. Designing Games. “O'Reilly Media, Inc.;” 2013. 1 p. 

