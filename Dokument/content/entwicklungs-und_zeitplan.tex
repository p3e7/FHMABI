\section{Entwicklungsprozess}
\label{cha:entwicklungs-und_zeitplan}
Um eine bessere Übersicht über den Zeit und Entwicklungsplan eines Videospiels zu erhalten, wird in diesem Kapitel das Konzept beschrieben, dass von dem Unternehmen Electronic Arts (EA) verwendet wird. EA ist eine im Jahre 1982 gegründete Kapitalgesellschaft, deren Hauptsitz in Redwood City in Kalifornien ist. Sie beschäftigt 9000 Beschäftigte und erwirtschaftete im Finanzjahr 2011 einen Nettoumsatz nach GAAP in Höhe von 3,6 Milliarden US-Dollar. EA entwickelt, veröffentlicht und vertreibt Videospiele. Zu den erfolgreichsten Spieleserien gehören Fifa, Mass Effect, Need for Speed und Die Sims.


User-centred design is a design philosophy which describes a prototype-driven software development process, where the user is integrated during the design and development process. The approach consists of several stages which are iteratively executed: Requirements analysis, user analysis, prototyping and evaluation. User- centred design is specified in EN ISO 13407 – Human Centred Design Processes for Interactive Systems (ISO 13407:1999 1999). This approach is also used for game design as described in (Fullerton et al. 2004) and a central topic at Microsoft Game Studios8. It contains three distinct development phases: conceptualization, prototyping and playtesting. The first phase typically involves the complete plan- ning such as identification of goals, challenges, rules, controls, mechanics, skill levels, rewards, story and the like (Pagulayan et al. 2003). These specifications are done by game designers and are put on record in game design documents.
The second phase–prototyping–is used to quickly generate playable content, which can be efficiently used to do playtesting and measure attributes such as user experience, the overall quality (commonly denoted as fun), the ease of use or the balancing of challenge and pace (Pagulayan et al. 2003).
To gather results for these variables a range of usability methods such as struc- tured usability tests (Dumas and Redish 1999) and rapid iterative testing and ev- aluation (RITE, Medlock et al. 2002) can be applied. Pagulayan et al. propose ad- ditional evaluation methods such as prototyping, empirical guideline documents or heuristics (Pagulayan et al 2003). We believe that especially heuristics can be a fast and cost-efficient but still effective and accurate evaluation method for user experience in games. Therefore we will present our own set of heuristics in sec- tions 5 and 7 and verify them by conducting an expert evaluation. Before that we will give a short introduction to heuristic evaluation as an expert-based usability approach.


\subsection{Idee und Konzeption, auch Preproduction}
%\label{sec:}
In dieser ersten konzeptionellen Phase, werden die ersten Ideen zur Spielmechanik, Hintergrundgeschichte sowie dem Artwork gesammelt und zusammengefasst. Diese Phase dauert, laut angaben von EA, in etwa zwei bis vier Monate. Während dieser Phase wird entschieden, ob die gesammelten Ideen überhaupt realisierbar sind und keinen zu hohen Kostenaufwand darstellen. Als weitere Überlegung wird bereits in dieser Phase überlegt, mithilfe welcher Spieleengine das spätere Spiel realisiert werden sollte. Auch bei dieser Überlegung wird wieder auf den Kostenfaktor für ggf. anfallende Lizenzkosten o.ä. geachtet. Da hier die Grundlegenden Gedanken zum späteren Spiel getroffen werden, wird diese Phase auch als Startphase bezeichnet und lässt sich auf jedes mögliche Szenario adaptieren.

\subsection{Finanzierung und Kostenplan}
%\label{sec:}
Auf Basis der Startphase werden ca. ein bis zwei Monate dafür verwendet, einen ausführlichen Kostenplan für die Entwicklung zu erstellen. Durch den Kostenplan soll ersichtlich werden, wie viel Geld für die Realisierung des geplanten Spiels benötigt wird. 
(Für das in dieser Arbeit zu entwickelnde Spiel wurde allerdings auf einen Kostenplan verzichtet, da sich dies als zu Komplex für den Umfang dieser Thesis herausstellte.)


\subsection{Entwicklungs- oder Produktionsphase}
%\label{sec:} 

Die eigentliche Entwicklung des Spiels findet in der dritten Phase statt, welche sich über einen Zeitrahmen von ca. ein bis zwei Jahren erstreckt, u.U. aber auch erheblich länger dauern kann. Die für diese Phase benötigte Zeit ist sehr schwer abzuschätzen und kommt auf das Ziel der Entwicklung an. Handelt es sich bei dem zu entwickelnden Spiel nur um eine Erweiterung eines bereits vorhandenen Spiels, kann dies die benötigte Zeit stark reduzieren wo hingegen eine komplette Neuentwicklung wesentlich viel mehr Zeit in Anspruch nimmt. 
Um die benötigte Zeit zu reduzieren, kann man die Ziele in kleine, in sich geschlossene, Teilbereiche einteilen (Divide and Conquer \cite{Choo:2002ty}) und an diesen Parallel arbeiten. Diese Unterteilung stellt sich allerdings zum Teil als schwierig heraus. So kann es z.B. bei aufeinander aufbauenden Teilbereichen zu Bottlenecks führen, die die Entwicklungszeit des Spiels wiederum verlängern kann. 
Ein weiterer Faktor, der die Entwicklungszeit eines Spiels beeinflussen kann, ist die ausgewählte Spielengine. Diese wird oftmals nicht selbst entwickelt, sondern von externen Quellen bezogen. Für kleine Unternehmen und Startups ist dies ein sehr wichtiger Schritt. Zum einen auf Grund des Kostenfaktors, zum anderen da sich an dieser Stelle entscheidet, ob das Spiel mit der gewählten Engine in Gänze entwickelt werden kann.
Die Entwicklung eines Spiels kann dabei in die drei Bereiche (\ref{sec:game_leveldesign},   \ref{sec:spielwelt_ki_physik} und  \ref{sec:erstellung_figure_objekte_audio_gui}) unterteilt werden. 

\subsubsection{Game-, Leveldesign und Drehbuch}
\label{sec:game_leveldesign}


Im ersten Teilbereich der Entwicklungsphase wird das Grundgerüst des Spiels entwickelt. Dies umfasst insbesondere das Leveldesign, sowie das verfassen der Story in Form eines Drehbuches. Ein weiterer Aspekt dieses Bereiches ist die Steuerung des Spiels, welche ggfs. für verschiedene Inputtypen geeignet sein sollte. Ein weiterer Aspekt ist der Entwurf des Drehbuchs für das Spiel, welches von einem Drehbuchautor entworfen wird. Der Umfang des Drehbuchs kann dabei, je nach Genre des Spiels, von der Komplexität sehr unterschiedlich sein.


\subsubsection{Spielwelt, Künstliche Intelligenz, Physik}
\label{sec:spielwelt_ki_physik}
Dieser Zweig kümmert sich um die Aufgaben an der Engine. Eine Aufgabe ist wäre das das Rendering der Spielwelt. Hier stellen sich Fragen wie, welche Shader werden eingesetzt oder in wie fern und wann die detailgetreueren Modelle ein und ausgeblendet werden müssen um nicht zu viel Performance zu verbrauchen. Des Weitern wird auch an der Künstlichen Intelligenz (im weiteren KI) und der Physik gearbeitet, je nachdem in wie weit sie zum tragen kommen oder in der Engine schon entwickelt sind. Dies variiert natürlich sehr stark. Da viele Spiele auch ohne Physik auskommen und auch nicht immer auf eine KI zurückgreifen müssen. 
                                                           
\subsubsection{Erstellung von Figuren und Objekten, Audio und Userinterface}
\label{sec:erstellung_figure_objekte_audio_gui}
Dieser Bereich ist zuständig für die Modellierung der 3D bzw. 2D Objekten, die später im fertigen Spiel verwendet werden sollen. Hinzu kommt die Kreation von Musik und gesprochenen Inhalten, die später, in den entsprechenden Situationen, dem Spieler des Spiels zu Ohren geführt werden. Nicht zu vergessen sind auch die ganzen Elemente aus denen sich die spätere Benutzeroberfläche, kurz GUI, zusammensetzt. 

\subsection{Parallel zu den Punkten \ref{sec:game_leveldesign} bis \ref{sec:erstellung_figure_objekte_audio_gui}}
%\label{sec:}
\subsection{Entwicklung Features und Download-Erweiterungen}
%\label{sec:}
Hier werden zusätzlich zu den normalen Bestandteilen des Spiels erste Zusatzinhalte entwickelt, die erst nach dem Erscheinungstermin des Spiels freigegeben und/oder verkauft werden. Es ist aber auch möglich, dass bei Spielserien die jährlich erscheinen ein oder mehrere Features über einen längeren Zeitraum entwickelt werden und erst nach 2 oder 3 Jahren in das aktuelle Spiel eingebaut werden.
\subsubsection{Zusammenfassung Marketing, Test und Motion Capturing}
%\label{sec:}
\begin{itemize}
\item Marketing und PR (Producer- und Entwickler-Videos, Screenshots, News)
\item Testphase und Fehlerbehebung (sechs Monate)
\item Motion Capturing und 3D Head Scanning (sechs Monate)
\end{itemize}

Die drei oben genannten Punkte sind natürlich ebenso wichtig wie vorhergehenden, da diese aber sehr zeitintensiv oder für unser kleines Team nicht zu bewerkstelligen sind werden sie nicht weiter im Detail beschrieben.

\subsection{Produktion und Verkauf (zwei Monate)}
%\label{sec:}
In diesem Schritt werden alle voran gegangenen Teile zu einer einheitlichen Version zusammengefügt. Wenn diese Fehlerfrei funktioniert, wird daraus die sogenannte Golden-Master oder Gold-Master. Diese kann dann an ein Presswerk geschickt werden oder evtl. weiter bearbeitet werden, damit sie auch als Download-Version verfügbar ist.
Parallel dazu wird auch noch eine Version entwickelt, die der Fachpresse zur Verfügung gestellt wird. Nebenher werden an dieser Stelle auch die Verpackung, die Anleitung erstellt und gedruckt und ein oder mehrere Werbespots gedreht. 

\subsection{Veröffentlichung von Patches, DLCs und Addons(Online)}
%\label{sec:}
Nach der Veröffentlichung eines Spiels stehen noch weitere Aufgaben an. Fast immer  muss mittels Patches an vielen Ecken und Enden des jeweiligen Spieles nachgebessert werden, da in den vorangegangenen Testphasen nicht alle Bugs entdeckt wurden. Nebenbei werden meist von kleineren Entwicklerteams zusätzlich noch DLC-Pakete entwickelt, die das Spiel in  seiner Gänze erweitern und so neue Level, Charaktere oder Items ins Spiel einfügen. Vor einigen Jahren, war es noch üblich Addons zu programmieren. Der Inhalt eines Addons ist im Grunde relativ ähnlich zu dem der DLC-Pakete. Der Umfang jedoch um ein vielfaches größer. So werden nicht nur zwei oder drei Level dem Spiel hinzugefügt, sondern eine ganze Kampagne, die aus bis zu 20 oder mehr Levels bestehen kann. 
\cite{Lorber:2011vl}

