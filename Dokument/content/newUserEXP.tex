\chapter{User Experience} 
%\label{sec:}
Im letzten Jahrzehnt ist der Begriff User Experience ein Modewort im Bereich der Mensch-Computer-Interaktion geworden. Dies hängt damit zusammen, das die Technologien sich im Bereich der Interaktion stark weiterentwickelt haben und somit nicht nur nützlicher und benutzbarer sondern auch mit der Mode gehen und als eine Art Schmuck gesehen werden. Dies steht im Kontrast zur Vergangenheit, in der besagte Technologien nur als Werkzeuge eingesetzt wurden.\\
Der Begriff der UX wurde in verschiedenen Fachkreisen schnell aufgenommen, wird aber an vielen Stellen als ungenau oder schwer definierbar betitelt. 
An andere Stelle wird er als ein Bereich zwischen Nutzbarkeit und positiven bzw. negativen Gefühlen bei Benutzung einer Technologie beschrieben. 
\cite{Forlizzi:2004dz}
Im Laufe der Zeit hat sich die Bedeutung des Begriffs UX, nach Meinung einiger Experten, sogar als Gegenbewegung zur vorherrschenden Usability entwickelt. Usability fokussiert im Gegensatz zur User Experience, wie bestimmte Aufgaben leichter bewältigt werden können.
Eine frühe Definition von UX setzt das Erlebnis einer Person in dem Moment, in dem es erlebt wird in den Mittelpunkt. Produktivität und Erlernbarkeit stehen dabei klar im Hintergrund.\cite{Whiteside:fe}
An anderer Stelle wird auf den Spaß in Verbindung mit der Usability hingewiesen.
\glqq We realize that many people will read this article as a joke. To this extent, we are the victims of our own
analysis: there are risks in being serious about fun. Still though, we continue to see, without humor, the prospect of a decade of research analysis possibly failing to provide the leverage it could on designing systems people will really want to use by ignoring what could be a very potent determinant of subjective judgments of usability – fun\grqq\
\cite{Carroll:1988hi}


\section{Standpunkte zur Bedeutung von UX} 
%\label{sec:}
Durch die Verschiebungen und Entwicklungen der Bedeutungen des Bereiches UX ist keine klare Definition möglich. Jedoch kristallisieren drei unterschiedliche Standpunkte heraus, die klar voneinander abgegrenzt sind und trotzdem insgesamt alle Ansichten beherbergen.  


\subsection{UX umfasst Usability} 
%\label{sec:}
UX ist ein weit gefächertes Konzept, dass neben anderen Aspekten auch Usability enthält. Im Vordergrund steht hier die Interaktion des Benutzers mit dem System und was er dabei Empfindet. Doch es wird zusätzlich noch ein größerer Rahmen um das Gesamtgebilde gespannt. Dieser besagt, dass UX schon vor dem Erwerb einer Software beginnt, sich über den Zeitraum der Nutzung erstreckt und zusätzlich noch andere Bereiche, wie den Support oder ähnliche Dienstleistungen beherbergt.

\subsection{UX ergänzt Usability} 
%\label{sec:}
UX geht mit Usability einher und fokussiert sich auf die emotionalen und experimentellen Aspekte. Zusätzlich fallen auch weitere Teilaspekte die nicht aufgabenorientiert sind und ursprünglich der Usability zugeschrieben wurden in den Bereich der UX.

\subsection{UX ist einer der Aspekte von Usability} 
%\label{sec:}
UX ist einer der drei Hauptbestandteile der Usability. Die ISO 9241-11 besagt, dass Usability sich aus drei Teilbereichen zusammensetzt, diese sind Effizient, Effektivität und Zufriedenstellung. Diese Sichtweise verknüpft die UX mit der Zufriedenstellung. 

\cite{Garrett:2003wx}
\textbf{WEITERE QUELLEN HINZUFÜGEN}



\section{Der Begriff Experience}
Da das Thema UX nicht nicht so leicht greifbar und definierbar ist 
Gemüt, Emotion und Stimmung

\glqq Everyone knows what an emotion is, until asked to give a definition\grqq (Fehr and Russel, 1984, p. 464). \cite{Zimmermann:2006uv}

%    In der Mensch-Technik Interaktion wird "User Experience" (Nutzungserleben) oft recht eng     als eine vom Produkt ausge- löste Bewertung des Benutzers verstan- den. Den Begriffen „Erleben“ und "Er- lebnis" wird diese Sichtweise nicht ge- recht, geht es doch dabei vielmehr um die Verknüpfung von Handeln, Fühlen, und Denken (in der Interaktion mit einem Produkt) zu einem Ganzen. "Experience Design" setzt sich nun zum Ziel, Erlebnisse gezielt zu schaffen (oder zumindest zu ermöglichen). Im Zent- rum steht also das zu gestaltende Er- lebnis und nicht mehr das Produkt. Dadurch ergibt sich eine Reihe von Herausforderungen an "experience design" und Gestalter. Zwei davon, nämlich die Kluft zwischen Bedürfnis und Produkt und die Rolle des zukünf- tigen Benutzers, werden diskutiert.


Zunaechst sind Erlebnisse per Definition subjektiv. Sie spielen sich \glqq im Kopf\grqq\ der Benutzer ab, die damit zum eigentlichen Qualitaetsmassstab werden. Gebrauchs- tauglichkeit wurde lange und wird auch heute noch als etwas Objektives ver- standen. Es geht eben eher um die Effi- zienz an sich als um ein Effizienzerleben (was sich auch trotz objektiv niedriger Effizienz einstellen kann, da objektive Merkmale und subjektives Erleben nur lose gekoppelt sind). Erleben ist also ein psychologisches Phaenomen, allerdings eben nicht im Sinne eines spezifischen psychologischen Prozesses, sondern als die unteilbare Repraesentation gerade bewusster Prozesse und Inhalte.


McCarthy und Wright (2004) benennen vier unterscheidbare \glqq Faeden\grqq\ der Erle- bens: das Sinnliche, dass Emotionale, das Raeumlich-zeitliche und das \glqq Zusammengesetzte\grqq\. Die ersten beiden betonen die Zentralitaet der Sinne und des Fuehlens; die letzten beiden betonen die Unteilbarkeit eines Erlebnisses und ganz besonders seine Dynamik (z.B. Karapanos et al., 2009). Natuerlich sind die Sinne und das Fuehlen untrennbar verknuepft mit Denken und Handeln. Er- leben kann also als eine Art innerer Kommentar verstanden werden – ein kontinuierlicher Strom aus Denken, Handeln, Fuehlen und Bewerten. Wird dieses Erleben zu einer in sich ge- schlossenen, bedeutungsvollen Episode zusammengefasst, entsteht ein Erlebnis (vgl. Forlizzi und Battarbee, 2004). Solche Erlebnisse geben unseren Handlungen Bedeutung, sie werden erinnert, kom- muniziert und wirken motivierend (oder abschreckend). Und natuerlich koennen interaktive Produkte eine Rolle in diesen Erlebnissen spielen: als Ausloeser oder Verstaerker des Erlebens.






\section{Aspekte} 
%\label{sec:}
TEX DATEI EINFÜGEN!!!\\
TEX DATEI EINFÜGEN!!!\\
TEX DATEI EINFÜGEN!!!\\


\chapter{Entwurf von Videospielen}
\label{cha:ext_tr_test}
Dieses Kapitel benennt und beschreibt die Eckpfeiler der Videospielentwicklung. Es wird gezeigt, mit welchen Prinzipien Spannung aufgebaut wird, in welchem Maß Graphik genutzt werden kann und auf welche Arten Erzählstrukturen eingesetzt werden können. Um diesen Überblick zu komplettieren wird gezeigt, wie ein Videospielentwicklungszyklus der Firma Electronic Arts aussieht.

\section{Basis Spielmechaniken}
\label{sec:basis_spielmechaniken}

%\section{Grundlegende Gedanken zur Spielmechanik}
%\label{sec:}
Der wichtigste und zugleich grundlegende Gedanke ist, wie andere Menschen dazu gebracht werden können Zeit in ein Spiel zu investieren. Dazu wird zunächst ein Blick in die Psyche der Menschen geworfen um zu entscheiden, was als Aspekt dienen kann. Das Spielen in der Kindheit ist eine Art lernen erwachsen zu sein. Ein blick in den Kindergarten zeigt, dass die meisten Kinder Fangspiele spielen oder bauen, z.B. im Sandkasten oder mit Bausteinen. Bei Jungen kommt es auch des Öfteren zu Kämpfen untereinander. \\
Alle Kinder spielen diese Spiele. Nicht nur ein- oder zehnmal, sondern jeden Tag und warum werden sie es nicht leid diese Spiele zu spielen.
Der wichtigste Punkt ist gewinnen macht Spaß. Jedoch darf die Herausforderung nicht fehlen. Genau dies ist es was die Kinder beim Spielen hält. Beim fangen, sowie beim Kämpfen werden auch die anderen Kinder immer geschickter, können besser ausweichen, werden schneller beim laufen. Beim Bauen müssen die Türme höher und stabiler werden. Trotzdem kann so etwas auch irgendwann langweilig werden, wenn jemand so viel besser als die anderen sind, dass keiner mehr mit ihm spielen will.\\

- Urtriebe\\
- Traning erwachsen zu sein \\
- nicht genau das was Videospiele sind \\


Ein andere Spielart entzieht sich diesem Prinzip und geht eine Ebene weiter, sog. Denkspiele oder Rätsel. Sie haben, anders als die zuvor genannten Spiele, die Eigenart, dass sie in den meisten fällen keinen Körperlichen Einsatz benötigen. Sie sprechen also einen völlig anderen Teil des Köpers an. Somit erhält man an diesem Punkt eine Zweiteilung zum einen die Spiele, die den Köper fordern und die Spiele, die den Geist vordern. Durch die Verknüpfung dieses Gedankengangs, kann man relativ schnell die Entwicklung der Menschheit nachvollziehen. Im Vergleich der physischen und geistigen Entwicklung lässt sich erkennen, dass die Menschheit im sich im physischen Bereich nicht so stark weiterentwickelt hat wie im geistigen Bereich. Dies zeigt sich auch an den Spielen, zu den oben genannten Basis Kategorien hat sich nicht kaum etwas hinzu entwickelt, wobei jedoch die Rätselspiele immer weiter entwickelt haben und immer komplexer geworden sind. Somit ergibt sich eine Zweiteilung die oft auch bei Kindern zu erkennen ist. So ist die eine Gruppe der Kinder eher dazu geneigt sich körperlichen bzw. sportlichen Aktivitäten zu widmen, während andere ihre Zeit lieber mit geistigen Aufgaben, wie lesen usw. verbringen. Da Gesellschaftsspiele, so wie Videospiele in der geistigen Kategorie angesiedelt sind, da man für sie kaum oder gar keine körperliche Kondition benötigt.

Im nächsten Schritt geht es darum das oben genannte Konzept auf ein Spiel zu übertragen, somit ergibt sich die erste Abstraktionsebene. Schach ist ein sehr altes Gesellschaftsspiel. Sicherlich gibt es noch andere Spiele die zuvor entstanden sind doch diese sind in dem Rahmen dieser Arbeit vernachlässigbar, da Schach eine erhöhte Komplexität beinhaltet und auch aktuell immer noch gespielt wird. Um Schach nicht genau im Detail zu beschreiben, lässt in kürze sagen, dass Figuren von 2 Spielern abwechselnd auf einem Spielfeld bewegt werden dürfen dabei versucht jeder Spieler Figuren vom anderen Spieler zu schlagen. Die Figuren dürfen jedoch nur nach bestimmten Regeln bewegt werden, dadurch entstehen situationsbedingte Vorteile von bestimmten Figuren über andere. Somit sollte jeder Spieler seine Figuren sets so positionieren, dass er einen Vorteil gegenüber dem anderen Spieler hat. Dies beinhaltet die oben genannten Urspieltriebe. Schaut man sich eine Runde im Detail an, so lässt sich erkennen, dass Spieler A versucht die Einheiten von Spieler B zu erreichen, bzw. zu fangen, er alternativ seine eigene Position auf dem Spielfeld verbessern kann, bzw. bauen oder Einheiten vom gegnerischen Spieler zu schlagen, also kämpfen. 

- Spielkonzepte mixen\\

Doch ist nicht jede Idee praktikabel und umsetzbar oder wird vielleicht durch andere Ideen obsolet. [Zitat] Bsp.: Gibt man einem Spieler die Möglichkeit sich in bestimmten Mustern über den Bildschirm zu bewegen und dem anderen Spieler die Fähigkeit großflächig schaden anzurichten, (Fußsoldat und Kampfjet) hat dieser einen unfairen Vorteil, der vielleicht auf den ersten Blick gar nicht auffällt. Daher muss man sich bei der Kombination verschiedener Spielelemente stets darüber im klaren sein, was man damit erreichen möchte...



\section{Tension-Release-Prinzip}
\label{sec:ext_tr_test}
Im ersten Teil dieses Kapitels wird die Bedeutung des Ausdrucks Tension-Release-Prinzips erläutert. Des Weiteren soll ein kleiner Ausblick darüber gegeben werden, an welchen Stellen dieses Konzept in Spielen verwendet wird.

\subsection{Veranschaulichung} 
%\label{sec:}
Die direkte Übersetzung von Tension-Release bzw. Tension and Release ist Anspannung und Entspannung (auch Erlösung). Dieses Konzept ist eines der wichtigsten Werkzeuge, um die Aufmerksamkeit des Spielers aufrecht zu erhalten. Im Grunde bedeutet es, dass der Spieler im dynamischen Wechsel vor schwierigere Aufgaben gestellt werden sollte und darauf folgend die anspannende Situation durch eine einfache Aufgabe aufgelöst werden soll. 
Schon die ersten Spiele haben dieses Konzept benutzt um Spannung zu erzeugen. Das lässt sich zum Beispiel auch an dem Spiel Schach feststellen. Bei diesem Spiel ist das Ganze jedoch auch noch an die Prämisse geknüpft, dass Spieler 1 einen Gegner auf einem ähnlichen Niveau hat. Dadurch, dass beide Spieler einen ähnlichen Wissensstand haben, stellen sie sich gegenseitig immer wieder vor schwierigere und leichtere Aufgaben. Dieses Schema lässt sich aber auch noch an einer weiteren Stelle finden und zwar bevor ein Spieler eine Figur des anderen Spieler schlägt, dabei ist der gesamte Weg dorthin mit Anspannung verbunden und im  Moment des Schlagens hört die Anspannung auf und wandelt sich in ein positives Gefühl, da ein angestrebtes Ziel erreicht wurde. 

\subsection{Fallbeispiel Tetris} 
%\label{sec:}
Dieses Prinzip findet sich so gut wie in allen erfolgreichen Spielen. Manchmal muss es jedoch genauer betrachtet werden. Nimmt man z.B. Tetris, so ist es bei diesem Spiel so, dass der Spieler aus Blöcken, die vom oberen Rand des Bildschirms herunterfallen richtig zu stapeln. Die Blöcke haben verschiedene Formen und können u.a. aussehen, wie ein L oder ein Z. Jedoch kann man diese ineinander schieben und somit den Bildschirm Reihe für Reihe füllen. Ist eine Reihe von Links bis Rechts aufgefüllt, verschwindet diese wieder. Dafür kriegt der Spieler Punkte und steigt im Laufe der Zeit Level auf. Je besser ein Spieler ist, desto höher wird das Level, somit fallen die Blöcke schneller nach unten. Dies deckt den Teil der Anspannung ab und jedes mal wenn es dem Spieler gelingt eine oder mehrere Reihen abzubauen, wird die Anspannung wieder gelöst, da der Abstand zur oberen Kante des Bildschirms mit jeder entfernten Reihe wieder vergrößert wird.

\subsection{Fallbeispiel Tamagotchi} 
%\label{sec:}
-Eher bedingt, nur bei den Evolutionsstufen\\
-Krankheiten\\
-Öfters verwendet in späteren Iterationen